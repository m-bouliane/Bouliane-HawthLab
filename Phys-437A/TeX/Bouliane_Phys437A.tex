\documentclass[12pt]{article}

\usepackage[margin=1.5cm, top=1cm]{geometry}
\usepackage{amsmath, amsfonts}
\usepackage[version=4]{mhchem}
\usepackage{mathrsfs}
\usepackage{graphicx}
\usepackage{subcaption}
\usepackage{hyperref}
\usepackage{cite}
\usepackage{listing}

\newcommand*{\figref}[2][]{%
  \hyperref[{fig:#2}]{%
    Figure~\ref*{fig:#2}%
    \ifx\\#1\\%
    \else
      \,#1%
    \fi
  }%
}

\begin{document}

\pagenumbering{roman}

\begin{titlepage}
    \begin{center}
        \vspace*{1cm}
 
        \textbf{Characterization and Optimization of the Photodetection Capabilities of an Inverse Photoemission Spectrometer}
             
        \vspace{1.5cm}
 
        \textbf{Michael Bouliane}
 
        \vfill
             
        A report submitted for partial fulfillment of the requirements of Phys 437A

        Source files and full sized images available at: \url{https://github.com/m-bouliane/Phys-437A}
             
        \vspace{0.8cm}
             
        Department of Physics and Astronomy\\
        University of Waterloo\\
        \today\\
             
    \end{center}
\end{titlepage}

\begin{abstract}
    The photodetection capabilities of an inverse photoemission spectrometer (IPES) were examined and characterized. The IPES system used two gas filled bandpass Geiger-M\"uller tubes, 
    one with a \ce{MgF_2} (left) entrance window and the other with a \ce{CaF_2} (right) window; both using acetone (act.) as a detection gas. Using LabVIEW, a custom macro was written to vary the applied
    high voltage to the detectors' anodes over a specified range, record the number of detection events using an MCA with the IPES electron gun powered on, and then again with the 
    gun off. The pressure of detection gas in each tube was varied and the scan was repeated for each new pressure value. Using this technique we examined how the tubes' performances
    changed as a function of their anode voltage and pressure and used this information to find a region of high stability and count rate. For the right detector we found optimal 
    performance at an acetone pressure of 3.7 torr and anode voltage of 750V, and 3.7 torr at 740V for the left detector. We also discovered several other regions of high counts below the tubes'
    breakdown voltages, providing alternatives to the ones reported should the need arise. Finally we plotted the breakdown voltage of each tube as a function of gas pressure, and 
    obtained results which qualitatively aligned with other reports in the literature. 
\end{abstract}

\clearpage

\tableofcontents
\clearpage

\listoffigures
\listoftables

\clearpage

\pagenumbering{arabic}

\section{Introduction}
    Inverse photoemission spectroscopy (IPES) is an experimental technique used in condensed matter physics to map the unoccupied electronic structure
    of a material. In an ultra-high vacuum environment, a beam of electrons of known energy is impinged on a sample. If the energy of the incoming
    electrons align with a state which lies above the fermi energy, then they will couple to that state. Detections come from photons released through
    radiative relaxation of these excited states, which are in the UV range for typical electron energies of 10-100eV. We are able to map the unoccupied states
    by recording the number of photon detections we obtain at a given incident electron energy. In order to obtain significant IPES results a spectrometer must meet strict requirements of both its photodectors and its electron source. Our previous efforts in 
    spectrometer characterization focused on its photodetection capabilities. In this report we detail our work to characterize the performance 
    of our electron source.

    The electron source, also referred to as an electron gun, must be capable of operating within at low electron energies while still producing a beam with a minimal spot size and 
    momentum spread. The current of electrons reaching the sample should also be as large as possible, equivalent to the gun having a high brightness. Achieving this is a highly 
    non-trivial due to the contradictory requests of being low energy but also high brightness\cite{stoffel1985low}. Emission current limitations in these type of guns are caused by the space charge effect, 
    where the cloud of low energy electrons that make up the beam repel incoming electrons back towards the source\cite{staib}. Commercially available 
    electron guns are capable of meeting these requirements and thus solve the problem of needing to carefully balance these two main considerations; however they introduce a 
    new problem in that one is now tasked with finding the operating conditions of the gun which allows it to do so. The electron gun in our spectrometer has 10 unique parameters
    which can each be varied by applying a voltage bias. So to obtain a high brightness beam which operates at low energies we need to determine a combination of these 10 
    voltages, an optimization problem that is made exceedingly difficult by this large parameter space. 

    Starting with the configuration recommended by our electron gun's manufacturer we measured the beam width and maximum current. The beam was found 
    to be insufficient for use in IPES, which was attributed to space charge limiting. Reducing the temperature of the gun's cathode allowed us 
    to remove a parameter from the optimization problem, and yielded much better results in subsequent tests. Further constraints were placed on the 
    possible configurations which could use for the gun, which again reduced the difficulty of the optimization problem. Eventually a set of parameters 
    was found, which produced a beam of gaussian profile with a small spot size and divergence angle, on par with what has been reported by other groups\cite{ipes,raj2004optimization,stoffel1985low,budke}.
    Subsequent calculations showed that this beam is able to resolve the first brillouin zone of copper. Our spectrometer is now in a condition where 
    only minor work needs to be carried out before we are able to produce meaningful IPES results.


\section{Inverse Photoemission Spectroscopy}

\subsection{The Inverse Photoemission Process}

A useful way to understand inverse photoemission is to think of it in the context the photoelectric effect. Photons incident on the surface of a sample with a sufficiently
large energy will lead to the liberation of electrons. The determining factor in this process is if the photon has an energy greater than or equal to the sample's work function, which 
is defined as the energy difference between the sample's fermi level and vacuum level. The fermi level is the highest energy occupied electronic state, and the 
vacuum level is the energy of a free electron above the sample's surface. Inverse photoemission is the dual of this process, where instead a beam of electrons incident on a sample 
leads to the emission of photons through radiative decay. 

The mechanism through which this occurs starts with the electron coupling to an unoccupied electronic state with an energy that matches its initial energy, the state must be unoccupied in 
order to satisfy the Pauli principle. This unoccupied state is of a relatively high energy compared to the fermi energy, meaning that the sample remaining in this configuration 
is energetically unfavourable. As a result the electron decays to a lower energy unoccupied state, and the energy difference between the two states is what determines the energy of 
the emitted photon. It should be noted that at this electron energy range the penetration depth of electrons into a sample is at its minimum\cite{penn1987electron},
making IPES a very surface sensitive technique.

There are two ways to use this process for spectroscopy. In the first, the electron energy is varied and only photons of a single frequency are detected. 
In the second, the electron energy is fixed and the full range of emitted photons is collected. These modes are known as isochromat and spectrograph respectively; our system operates in the isochromat 
mode so we will limit the scope of further discussion to reflect this. 

\subsection{Energy Conservation in Inverse Photoemission}

In isochromat mode we vary the energy of electrons and record the intensity of photons with a set frequency $\omega_0$, thus the electron can only decay to states which meet the requirement that:

\begin{equation}
    E_i = \hbar\omega_0 + E_f
\end{equation}

Which means that if we know $E_i$ and $\omega_0$ then we can determine the energy of the unoccupied state $E_f$. So IPES allows us to map all of the $E_f$ states by varying $E_i$ and 
recording photon intensity. In general determining $E_i$ is a difficult task, since it is defined as:

\begin{equation}\label{vg}
    E_i = eV_g + \Phi_g
\end{equation}

where $V_g$ is the voltage applied to the electron gun's cathode, and $\Phi_g$ is the work function of the cathode. The difficulty comes from the fact that we don't know $V_g$ to great 
precision, and don't have a value for $\Phi_g$ at all. We can overcome this by measuring $E_f$ relative to the fermi energy, so for example if we get a peak in our spectrum at 
$E_f = 5\textrm{eV}$, then we know that there is an unoccupied state at an energy of 5eV above the fermi level. As we will later see however, 
this approach is insufficient if we want to perform what is called an angle resolved measurement on the sample. In this case we need to know the 
momentum of the electrons reaching the sample, and doing so requires a knowledge of the kinetic energy of the electrons which depends on $E_i$. 

\subsection{Momentum Conservation}

At the scale of an electron incident on a macroscopic sample, there is essentially infinite translational symmetry in the directions parallel to the sample's surface.
This symmetry means the electron's momentum in those parallel directions must be conserved as it changes from being a free electron to being bound 
in one of the sample's energy levels. This adds a further requirement that the momentum of the states to which the electron couples and decays into must match its initial parallel momentum. 
If we consider the kinetic energy of a free electron\cite{ipes}, just before it is bound to the sample we have:

\begin{equation}
    E_k = \frac{\hbar^2 k^2}{2m_e}
\end{equation}

Solving for the magnitude of the electron's momentum and we get:

\begin{equation}
    k = \frac{\sqrt{2m_e E}}{\hbar}
\end{equation}

If we define $\theta$ as the angle between the beam's axis and the samples surface, then the parallel component of the momentum is given by:

\begin{equation}\label{kpar}
    k_\parallel = k\sin{\theta}
\end{equation}

So by changing the incidence angle of the electron beam we are able to select states with the same $E_i$ value but with a different $k_\parallel$, letting us probe much more of 
the band dispersion. This technique is known as angle- or k-resolved inverse photoemission spectroscopy (KRIPES) and is one of the biggest selling points of the IPES method. 

\subsection{The Contact Potential}

Similarly to $E_i$, determining a value for $E_k$ is equally difficult since it is defined as:

\begin{equation}
  E_k = E_i - \Phi_s
\end{equation}

Where $\Phi_s$ is the work function of the sample undergoing inverse photoemission. If we insert equation \eqref{vg} we find that:

\begin{equation}
  E_k = eV_g + \Phi_g - \Phi_s
\end{equation}

We define the contact potential, $\Delta$, to be:

\begin{equation}\label{delta}
  \Delta = \Phi_s - \Phi_g
\end{equation}

Which tells us that the kinetic energy of the incident electrons is given by:

\begin{equation}\label{Ek}
  E_k = eV_g - \Delta
\end{equation}

Since $\Delta$ is dependent on the work function of the sample, its exact value will change for every sample we study with IPES. Determining a value for $\Delta$ can thankfully be
done by following a simple procedure\cite{mcmahon_2012}. With the electron beam incident on a sample with a known energy, we apply a bias voltage to the sample and record the 
resulting sample current. With an increasing sample bias, the current should remain mostly constant until reaching a threshold value where the current rapidly drops to zero. Keeping 
the sample bias voltage and gun cathode voltage in mind we consider the required $V_g$ to overcome the potential barrier created by having $\Phi_s > \Phi_g$:

\begin{equation}
  V_g = \frac{\Phi_s - \Phi_g}{e}
\end{equation}

If we now apply an arbitrary bias voltage to the sample the cathode voltage needs to increase to:

\begin{equation}
  V_g = V_s + \frac{\Phi_s - \Phi_g}{e}
\end{equation}

Inserting equation \eqref{delta} and solving yields:

\begin{equation}\label{vgvs}
  \Delta = e(V_g - V_s)
\end{equation}

Thus using the cathode voltage and the threshold bias voltage we recorded, we can determine the contact potential and from that determine the kinetic energy of the electrons in beam.

\section{The Electron Gun}

\subsection{Thermionic Emission}

The electron gun used in our spectrometer is a thermionic emission type. In this design, a heated cathode is used as the source of electrons. The 
high temperature of the filament allows for electrons in the cathode to be thermally excited to its vacuum level. Low work function cathodes 
are chosen to reduce the operating temperature needed to excite the same amount of electrons. In the case of our electron gun, this is achieved by 
using a BaO coated filament which has a lower work function than the same filament without a coating\cite{kimballphysics}. The current density 
emitted from the cathode at a temperate $T$ and work function $\Phi$ is given by Richardson's Law:

\begin{equation}\label{richardson}
  J = A_G T^2 \mathrm{e}^{-\frac{\Phi}{kT}}
\end{equation}

Where $A_G$ is the cathode material's Richardson constant. From this we see that emission current is inversely proportional to work function, and 
proportional to cathode temperature. Thus we can obtain a comparable current to a hot, high work function cathode using a colder and lower work function 
cathode. A gun operating at these reduced temperatures will have a longer cathode lifetime, and is generally referred to as being in the temperature limited 
mode.

\subsection{Space Charge Effect}

When working with low energy electron beams, the space charge effect has a much greater impact on maximum achievable beam currents than 
high energy counterparts\cite{staib,stoffel1985low,raj2004optimization}.
As electrons are emitted from the filament, they form a space charge which will eventually form the beam. As the charge of this distribution increases, 
its repulsive force can send electrons back to the filament. It can further prevent electrons from leaving the filament in the first place. The current 
density of a space charge limited cathode is given by Child's Law:

\begin{equation}
  J = K \frac{V_d^{3/2}}{d^2}
\end{equation}

Where K is a constant, $V_d$ is the potential difference between the cathode and anode, and $d$ is the cathode to anode distance.
We notice that unlike Richardson's law, there is no longer a temperature dependence for our current density, meaning that a space charge limited
beam is resistant to temperature fluctuations. However, space charge limiting only becomes a significant factor at high cathode temperatures, 
after the gun moves out of the temperature limited mode.

\clearpage
\subsection{Electron Gun Construction}

\begin{figure}[h!]
  \centering
  \includegraphics[width=0.85\linewidth]{../Assets/Gun diagram.png}
  \caption{Block Diagram of thermionic emission type electron gun for use in low energy applications\cite{gina_2012}}
  \label{fig:egun}
\end{figure}

Figure \ref{fig:egun} shows a typical design for low energy electron guns which use a heated cathode. The ground of the electron gun is connected 
to the supply which controls electron energy, meaning that all potentials are going to be floated by whatever potential is selected for electron energy. 
The sample and gun assembly share the same ground, meaning that the gun's potentials are also floating with respect to the sample. 

The filament supply voltage controls the temperature of the cathode, and is limited to a range of 0V to 5V. The temperature is increased through ohmic heating and an emission 
current is generated through thermionic emission. The electrons are accelerated through the gun by the extraction potential (labeled first anode in figure \ref{fig:egun}),
which is limited to a range of 0V to 100V. After being pulled from the cathode the electrons pass through the grid, which controls the brightness of the beam. 
It can be biased with a potential between -15V and 15V, a sufficiently negative grid voltage can suppress the emission of electrons entirely. The shaping
of the beam is done by the focusing electrodes. Our gun has two separate electrodes which can be independently biased in a range determined by the electron 
energy and extraction potential. Finally, the beam can be steered in the two directions perpendicular to its axis by using two parallel plate deflectors. 
This can be used to realign the beam towards the sample if it is found to be deflecting off target. 

The goal in characterizing an electron gun is to find a combination of all of these parameters which meet the strict requirements of inverse 
photoemission. A notable thing to balance is the cathode temperature and extraction voltage, as it controls if the gun is operating in the temperature limited 
or space charge limited regime. In the temperature limited regime we have longer filament lifetimes and a potential for larger sample currents, but 
the emission current is highly dependent on cathode temperature. Space charge limiting provides us stability in emission current but at the cost of 
a hot filament with a much lower lifetime and maximum emission. Best results are obtained when there is some balance between the two, yielding a stable and sufficiently high 
emission current, while also prolonging cathode lifetime\cite{stoffel1985low,staib}. Other considerations include keeping the focus potential to roughly the same order as the extraction 
potential, which minimizes the amount of accelerating and decelerating an electron in the beam undergoes, which is more likely to provide a well 
behaved beam\cite{stoffel1985low,raj2004optimization}.
\section{Experimental Design}

\subsection{Sample Preparation}
Before performing any kind of measurements, we first need to ensure that the sample's quality is as high as possible. This means it should be as close to atomically smooth and free 
from adsorbed contaminants as we can manage. This is achieved by using a combination of Argon ion (\ce{Ar+}) bombardment and annealing of the sample. Progress is qualified by examining 
the low energy electron diffraction (LEED) pattern obtained from the sample. 

The sample used for every trial in this experiment is a 1cm $\times$ 1cm $\times$ 1mm Cu(111) metal square, which was stored in atmosphere in a desiccator . It was transferred into the IPES system 
by means of a sample garage in a load-lock chamber. Once the load-lock was brought down to vacuum the sample was then seated in a holder within the preparation chamber. This holder
is height adjustable and allows for the sample to be aligned with the \ce{Ar+} beam for etching; the holder also contains a button heater which is powered from an external lab bench 
supply, and is used for annealing the sample. \figref{prepchamber} shows the Cu sample seated in the sample holder (centre of viewing window), aligned with the beam from the \ce{Ar+}
gun (top left). 

\begin{figure}[h!]
    \centering
    \includegraphics[scale=0.08]{Figs/prep.jpg}
    \caption{Prep chamber set up for \ce{Ar+} etching of Cu sample}
    \label{fig:prepchamber}
\end{figure}

With the sample loaded and aligned, the ion gun was turned on and set up. First the ion current was set to 20mA, then the beam energy was set to 1kV. Once the gun was up to temperature
and outgassing from the heated filament had stopped, Ar could be introduced to the gun to be ionized and accelerated. A needle valve controls the flow of gas into the gun, and it is 
connected to a filter to remove trace gases as the Ar flows from the main cylinder. The pressure of gas was increased until it settled at approximately $1\times10^{-5}$ torr.
The sample was left in this configuration for a full day, after which the gas was turned off and the gun was powered down. The sample was then transferred to the main chamber once the 
pressure of residual Ar in the prep chamber had decreased. Once in the main chamber, it was aligned so that a LEED pattern could be obtained. The sample was then transferred back to the prep chamber and seated in the sample holder for annealing. 

To anneal the sample the button heater was connected to a lab bench power supply by means of a feedthrough in the prep chamber. The current on the supply was set to 4.5A and the temperature
of the sample was monitored using a K-type thermocouple in contact with the heater stage. The temperature of the stage and sample rose slowly over the period of a day, finally 
settling at 450$^\circ$C. Once this temperature was achieved the sample was left to anneal for another full day. The goal of annealing is to heat the sample enough that atoms on the 
surface become mobile allowing for atoms on peaks in the sample to fill in valleys, leading to a smoother surface\cite{robinson2012argon}. The process of obtaining a LEED pattern was 
repeated once the sample had cooled to room temperature, a comparison of LEED patterns for etching with no annealing, and etching with annealing can be seen in \figref{LEED}. The sample
was then left in the main chamber for the next steps of the experiment. 

\begin{figure}[h!]
  \centering
  \subcaptionbox{LEED pattern obtained for Cu(111) sample cleaned using only \ce{Ar+} etching. Spots which are not part of $C_6$ symmetric pattern indicate defects in the surface.}[0.35\linewidth]{\includegraphics[scale=0.07]{Figs/etched.png}}
  \subcaptionbox{LEED pattern for \ce{Ar+} etched and annealed Cu(111). Only the expected $C_6$ pattern is visible indicating a smooth surface free of contaminants.}[0.35\linewidth]{\includegraphics[scale=0.07]{Figs/annealed.png}}
  \caption{Comparison of LEED patterns obtain for Cu(111) during different phases of sample preparation. The only pattern visible at this energy should be a hexagon, reflecting the 
  hexagonal shape of the (111) plane of an FCC lattice.}
  \label{fig:LEED}
\end{figure}

\subsection{Experimental Setup}

Within the main vacuum chamber of the system is where IPES takes place. A sample is loaded into a motorized sample holder with the ability to translate in X, Y, and Z directions 
and to rotate about the central (Z) axis of the chamber. Once mounted the sample is moved to be placed in front of the output of the electron gun. The photodetector tubes, mounted to 
ConFlat bellows, are then brought close to the sample to maximize the solid angle made up by the detectors. The tubes are mutually connected to a gas manifold that allows control 
over whether a tube is pressurized (through the use of valves), and the extent of the pressurization. The manifold is connected to a small bottle of spectroscopy grade acetone, and
a turbopump cart; by opening the needle valve to the acetone bottle the pressure of the tubes can be increased, and by opening a valve to the pump cart it can be decreased allowing 
for fine control over the pressure of the system.

The detectors are connected to a Cremat charge sensitive preamplifier kit\cite{CR-150-R5}, which contain a CR-110-R2.1 preamplifier chip\cite{CR-11X}. This chip has a gain of 1.4V/pC, and is the 
largest gain available from Cremat, it was chosen because of the small charge pulses that come from IPES detections. The connection from the preamp to the detector is done through a 
safe high voltage (SHV) feedthrough which is connected directly to the anode wire. The preamplifier is responsible for converting the current pulse from the anode, caused by the 
Townsend avalanches, into a voltage pulse with signal strength proportional to the input current. The preamp contains an SHV input which is used to bias the detector, and 
an a coax output for the voltage pulse. The output signal is fed into a Cremat Guassian shaping amplifier\cite{CR-160-BOX-R4}, which converts the step-function voltage pulse from 
the preamplifier into a Guassian signal. The shaping amplifiers for both detectors use a CR-200-500ns-R2.1 shaping module, which has a 500ns shaping time and yields a Gaussian pulse
with a FWHM of 1.2$\mu$s\cite{CR-200-X}. A comparison of the two pulses can be seen in \figref{pulses}, which shows the step-function pulse from the preamplifier in yellow, and the 
resulting Gaussian pulse from the shaping amplifier in blue. 

\begin{figure}[h!]
  \centering
  \includegraphics[scale=0.15]{Figs/pulses.png}
  \caption{Preamplifier (yellow) and shaping amplifier (blue) pulses shown on an oscilloscope}
  \label{fig:pulses}
\end{figure}

The output pulses from both shaping amplifiers are fed into two ADC converter inputs on a FAST ComTec multichannel analyzer. The MCA is set up to detect Gaussian input pulses above a certain threshold 
value, which from experimentation was set to 30mV. Most signals with pulse height less than 30mV were found to just be electronic noise cause by the addition of a resistor to the preamplifier
to prevent damage from over-current due to tube breakdown. The MCA keeps a running count of how many pulses above this threshold it detects in a given amount of time, and in addition 
automatically produces a pulse height distribution. The number of counts from the MCA is then finally sent to a computer using LabVIEW. 

\subsection{Procedure}

The general procedure for this experiment is based on the work done by Banik and Shukla, who examined what combination of acetone pressure and anode voltage 
would maximize counts and minimize tube breakdowns\cite{banik2005optimal}. Our experiment started by evacuating the tubes of acetone, which was done because the energy involved from tube breakdown is large
enough to crack the acetone rendering it useless for detecting photons\cite{banik2005optimal}. The tubes were then filled with acetone to the desired pressure. Next the electron gun and 
high voltage power supply were powered on, with 5V being passed to the gun's filament to provide a source of electrons using thermionic emission. A LabVIEW virtual instrument was used 
to control the various components of the experiment. The following is an excerpt from the command file written for the LabVIEW program:

\begin{center}
  \begin{verbatim}
    set;va;300.0;
    set;vb;300.0;
    set;en;15.0;
    measure;gm1;
    measure;gm2;
    measure;6487c;
    set;en;0.0;
    measure;gm1;
    measure;gm2;
    measure;6487c;
  \end{verbatim}
\end{center}

The program starts by setting the bias voltage used for both detectors, in this case using a value of 300V. Next the electron gun is set to emit electron with an energy of 15eV, 
which was chosen simply because it gave a high count rate for the Cu sample used. The program is told to wait 3 seconds before executing the next command which is done to allow the emission of photons from the sample to stabilize and give an accurate impression 
of the count rate. When the waiting period is over the program then runs the MCA for a predetermined amount of time, which was either 30, 15, or 5 seconds, and record the number of counts
detected during that time. Finally the current of the sample is measured using a picoammeter and is recorded. The emission of electrons is then turned off, and the same process of 
measuring counts from the system is repeated. The reason for turning the gun off is to observe how many detections can be attributed solely to breakdown from the tube. The voltage 
applied to the tubes is increased by the program and the measurements start over. This process is done until the anode voltage reaches a predefined upper limit, at which time the tubes 
are evacuated and refilled to the next pressure value. 
\section{Results}

\figref{paramLGM} shows the counts from the left GM tube over the full range of parameter space explored, with the electron gun both on and off; \figref{paramRGM} shows the same 
data for the right GM tube. 

\begin{figure}[h!]
    \centering
    \subcaptionbox{Count rate from left GM tube as a function of anode voltage and acetone pressure. Electron gun off}[0.49\linewidth]{\includegraphics[scale=0.65]{Figs/LGMGunOff.jpg}}
    \subcaptionbox{Count rate from left GM tube as a function of anode voltage and acetone pressure. Electron gun on}[0.49\linewidth]{\includegraphics[scale=0.65]{Figs/LGMGunOn.jpg}}
    \caption{Performance of left GM tube over entire parameter space of pressures and voltages}
    \label{fig:paramLGM}
\end{figure}

\begin{figure}[h!]
    \centering
    \subcaptionbox{Count rate from right GM tube as a function of anode voltage and acetone pressure. Electron gun off}[0.49\linewidth]{\includegraphics[scale=0.65]{Figs/RGMGunOff.jpg}}
    \subcaptionbox{Count rate from right GM tube as a function of anode voltage and acetone pressure. Electron gun on}[0.49\linewidth]{\includegraphics[scale=0.65]{Figs/RGMGunOn.jpg}}
    \caption{Performance of right GM tube over entire parameter space of pressures and voltages}
    \label{fig:paramRGM}
\end{figure}

For both tubes we see that there is an onset of counts earlier with the electron gun on than when the gun is off, these are indicative of detections of photons from the IPE of the copper sample.
By examining at what potential counts start to appear when the electron gun is off, we can determine the location of the breakdown voltage for each tube at each tested pressure. This 
gives us a new range of parameter space to explore, where the vast majority of counts can be attributed solely to IPE detections. The breakdown voltage was determined manually by examining plots 
of counts against anode voltage for each tube at each pressure. An example of such a plot can be found in \figref{VB}, and repeating for the remaining pressures yields \figref{breakdown}.

\begin{figure}[h!]
    \centering
    \includegraphics[scale=0.8]{Figs/example.jpg}
    \caption{Example of counts data as a function of voltage when electron gun is turned off. For this pressure, 2.82 torr, the breakdown voltage was determined to be 738V}
    \label{fig:VB}
\end{figure}

\begin{figure}[h!]
    \centering
    \includegraphics[scale=0.8]{Figs/VB.jpg}
    \caption{Breakdown voltages as a function of pressure for both left and right GM tubes}
    \label{fig:breakdown}
\end{figure}

With the breakdown voltages determined we can then truncate \figref{paramLGM} and \figref{paramRGM} to determine the regions of maximum counts and stability. A subtraction was 
also performed of counts while the gun is off from counts while the gun is on in order to provide a better estimate of the number of genuine detections at each configuration. 

\begin{figure}[h!]
    \centering
    \subcaptionbox{Counts for left GM tube below breakdown voltage}[0.49\linewidth]{\includegraphics[scale=0.65]{Figs/LGMFinalNorm.jpg}}
    \subcaptionbox{Counts for right GM tube below breakdown voltage}[0.49\linewidth]{\includegraphics[scale=0.65]{Figs/RGMFinalNorm.jpg}}
    \caption{GM tube performance below breakdown voltage}
    \label{fig:results}
\end{figure}

Using this we determined optimal operating conditions for each tube to be:

\begin{table}[h!]
    \begin{center}
    \begin{tabular}{ccc}
    Tube & \multicolumn{1}{c}{\begin{tabular}[c]{@{}c@{}}Pressure\\ (torr)\end{tabular}} & \multicolumn{1}{c}{\begin{tabular}[c]{@{}c@{}}Anode Voltage\\ (V)\end{tabular}} \\ \hline
    Left & 3.70 & 740 \\
    Right & 3.70 & 750
    \end{tabular}
    \caption{Optimal conditions for left and right tubes }
    \label{tab:ideal}
    \end{center}    
\end{table}

We can consider a small subsection of these plots near the optimal region to see how the identified parameters compare to neighbouring regions. 

\begin{figure}[h!]
  \centering
  \subcaptionbox{Counts for left GM tube around identified optimal region}[0.49\linewidth]{\includegraphics[scale=1]{Figs/LGM Waterfall.jpg}}
  \subcaptionbox{Counts for right GM tube around identified optimal region}[0.49\linewidth]{\includegraphics[scale=1]{Figs/RGM Waterfall.jpg}}
  \caption{Plots of counts for left and right GM tubes near their optimal operating conditions}
  \label{fig:waterfall}
\end{figure}

\section{Discussion}

Since the results are entirely dependent on determining the regions of parameter space which are stable (no tube breakdowns), we should begin by seeing if our result for breakdown 
match with what has been reported in literature. From the insert plot in \figref{litcompare}, we see that the results in our experiment match qualitatively with those obtained from Banik et al. early on, 
but diverge when our breakdown voltage reaches a maximum before decreasing and ultimately plateauing. While not explicitly stated in their paper, our tube's construction is almost certainly 
different from theirs, with the most important difference being the radius of the anode wire and inner diameter of the tubes. The most likely conclusion is that due to the geometry 
differences we are exploring more of the effects of tube breakdown over a similar pressure range. Since the tubes are completely independent systems once they are sealed from the 
gas manifold and have been designed to the same specifications, the fact that their breakdown voltages line up so strongly corroborates the notion that we are simply seeing more of 
the tubes' breakdown effects than Banik et al. 

\begin{figure}[h!]
  \centering
  \includegraphics[scale=0.6]{Figs/litcompare.png}
  \caption{Main: Counts as a function of anode voltage for varying acetone pressure. Insert: Breakdown voltage as a function of pressure}
  \label{fig:litcompare}
\end{figure}

We see the maximal count rates for both tubes and how they compare to the rates of neighbouring regions in \figref{waterfall}. Though difficult to tell because of how much larger the 
results for 3.7 torr are (full size images available online), we see that each of the curves have the same shape: an initial large region of no detections, a sharp increase in 
count rate, then a maximum value before a sharp decrease after the breakdown is reached. This is exactly what we see in \figref{litcompare}, showing more qualitative agreement in how 
our GM tubes operate compared to what has already been published. 

In addition to the strong agreement in breakdown voltages for the two tubes, we see that they both achieve maximal count rates at 3.7 torr and within 10V of each other. We can use these
results to determine how the window material affects count rate. We know that the left tube uses \ce{MgF2} and the right tube uses \ce{CaF2}. The transmission threshold of \ce{MgF2} 
is reported to be 10.97eV\cite{lipton2002photon}, and 10.2eV for \ce{CaF2}\cite{funnemann198610}; further the photoionization potential of acetone is reported to be 9.7eV\cite{funnemann198610}.
This means we have a larger bandpass for the left tube than the right tube, i.e the right tube's window will absorb more photons than the left tube's and so we should expect less counts
from the right tube. Comparing the results shown in \figref{waterfall} we see that the left tube has a maximum count rate which is approximately 10 times larger than that of the right tube, 
which is exactly what we predicted based on the bandpass of the two detectors. Despite our desire for a high count rate, this is simply because IPES yields relatively few photons 
that we can detect; the energy resolution of the overall system is dependent only on the energy resolutions of the detectors and electron beams. These initial results however are 
not enough to make a determination on which window material we should use. While they show that the count rate with a smaller bandpass is still entirely useable for IPES, we need to 
consider the energy resolutions of both detectors in order to have a full picture on how one detector configuration compares to the other. This is commonly done in literature by 
performing IPES on a sample of polycrystalline gold or silver\cite{funnemann198610}\cite{maniraj2011high}, since their density of states is linear near the fermi energy. The resulting 
spectrum should be step function-like about $\varepsilon_F$ and the width of the function is indicative of the energy resolution. While this is an important experiment to perform in the coming term, it 
is important to note that it may have no impact on the detector setup since we may find that the current configuration is acceptable for obtaining accurate IPES results.

Obtaining pulse height distributions are another important component for understanding how the tubes operate, though we sadly ran out of time in the fall term before we could perform it.
The procedure involves simply filling the tubes to 3.7 torr, setting the appropriate anode voltages, and then letting the MCA run for long enough to obtain a statistically significant amount 
of detections. The pulse height data is automatically recorded by the MCA so we would simply need to fit an appropriate distribution to it and extract the mean pulse height and standard deviation. 
The mean pulse height (in volts), can be used in combination with the gain values of the electronics in order to determine the number of electrons involved in a given detection. A 
GM tube operating in the proportional count region will have detections involving on the order of $10^4$ electrons, whereas a detector operating in the geiger region will have 
detections involving on the order of $10^9$ electrons\cite{knoll2010radiation}. While both proportional and geiger regions are acceptable for IPES, the geiger region is less stable 
for the detector over a long period of time. This means needing to refresh the acetone more often than if the detector is operating in the proportional count region, and a greater 
risk to the electronics due to the larger current pulses. Therefore if we determine that at the parameters outlined in table \ref{tab:ideal} have the detectors operating in the 
geiger region we may need to consider a new region which has a lower count rate but is in the proportional count region and thus further from tube breakdown. 

\section{Conclusion}
We investigates the photodetection capabilities of two gas filled bandpass detectors for use in inverse photoemission spectroscopy. Maximum count rates were found at an acetone 
pressure of 3.7 torr and an anode voltage of 750V for the tube with a \ce{CaF2} entrance window, and 3.7 torr at 740V for the tube with a \ce{MgF2} entrance window. We further determined
that due to the smaller bandpass of the \ce{CaF2}, the count rate obtained at this optimal operating was significantly smaller than the tube with a \ce{MgF2} entrance window with a 
larger energy bandpass. Due to time constraints we were unable to determine the energy resolution of the tubes at these operating conditions, though we proposed a method to do so 
in the new term which involves performing IPES on polycrystalline gold or silver and looking at the broadening near the fermi energy. We were also unable to examine the pulse height 
distributions of the two tubes, which can be used to find an average pulse height, and thus determine the operating region of the tubes based on the number of electrons in each detection. 

We have also shown that the two tubes, which are completely isolated systems and constructed to identical specifications, showed strong agreement in what parameters yielded the most 
counts and at what voltages breakdown would occur. These results also agreed qualitatively with those reported by Banik et al., whose experiment inspired the work done for this report. 
This leads us to conclude that we have successfully examined the relevant ares of parameter space for our tubes, their breakdown voltages, and thus the optimal conditions at which we should 
be operating. 

Building on these results in the next term, we hope to finish our analysis of the GM tube performance and then shift focus to optimizing the performance of the electron gun set up. 
The parameter space of the gun is much larger than the tubes, and extracting relevant information like beam width is a complicated procedure so we hope that the remaining work on the GM 
tubes can be completed very quickly. We also hope to finally obtain good IPES spectra from various samples kept in the lab which agree with those reported in literature. Most important 
among these being polycrystalline gold and silver, a full scan of Cu(111), and the image-potential states of Cu(100).

\clearpage
\section{Acknowledgements}
I'd like to thank my supervisor, Prof. David Hawthorn, for providing me the opportunity to learn and grow as a researcher. I've been working with him for 8 months now and have learned
so much about condensed matter physics, experimental skills, and how to think like a physicist. I'd also like to thank all of the members of the Hawthorn group, who were always there 
to offer advice when I got stuck, celebrated my successes, and shared the amazing projects they're all working on. Finally thank you to those of you who dedicated your time to read 
my report, I appreciate your efforts greatly. 
\clearpage
\bibliography{citations}
\bibliographystyle{apalike}

\end{document}